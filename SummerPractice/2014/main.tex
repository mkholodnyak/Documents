\documentclass[a4paper,12pt]{article}

\usepackage[T2A]{fontenc}
\usepackage[utf8]{inputenc}
\usepackage[english,russian]{babel}

\usepackage{amssymb,amsfonts,amsmath}

%\usepackage{algorithmicx}
\usepackage{algorithm}
\usepackage{algpseudocode}

\renewcommand{\algorithmicrequire}{\textbf{Input:}}
\renewcommand{\algorithmicensure}{\textbf{Output:}}

\usepackage{geometry}
\geometry{left=2cm}
\geometry{right=2cm}
\geometry{top=2.5cm}
\geometry{bottom=2.5cm}

% Установка своего колонтитула
\usepackage{fancyhdr}
\pagestyle{fancy}

% Список литературы через точку.
\makeatletter
\bibliographystyle{unsrt}
\renewcommand{\@biblabel}[1]{#1.}
\makeatother

% Для вставления ссылок на веб-ресурсы
\usepackage{url}

% Кликабельные референсы
\usepackage{hyperref}

% Для больших греческих букв (в стандартном наборе не все)
\usepackage{upgreek}

% Настройка border для таблиц
\usepackage{booktabs}

% Векторная графика
\usepackage{tikz}
\usetikzlibrary{arrows}

% utf8 + uppercase
\usepackage{textcase}

% Позволяет подключать pdf к файлу
\usepackage{pdfpages}

% Вывод исходного кода
\usepackage{listings}

% Именованные референсы
\usepackage{nameref}

%%%%%%%%%%%%%%%%%%%%%%%%%%%%%%%%%%%%
% Default fixed font does not support bold face
\DeclareFixedFont{\ttb}{T1}{txtt}{bx}{n}{12} % for bold
\DeclareFixedFont{\ttm}{T1}{txtt}{m}{n}{12}  % for normal

% Custom colors
\usepackage{color}
\definecolor{deepblue}{rgb}{0,0,0.5}
\definecolor{deepred}{rgb}{0.6,0,0}
\definecolor{deepgreen}{rgb}{0,0.5,0}

% Python style for highlighting
\newcommand\pythonstyle{\lstset{
language=Python,
basicstyle=\ttm,
otherkeywords={self},             % Add keywords here
keywordstyle=\ttb\color{deepblue},
emph={MyClass,__init__},          % Custom highlighting
emphstyle=\ttb\color{deepred},    % Custom highlighting style
stringstyle=\color{deepgreen},
frame=tb,                         % Any extra options here
showstringspaces=false            % 
}}


% Python environment
\lstnewenvironment{python}[1][]
{
\pythonstyle
\lstset{#1}
}
{}

% Python for external files
\newcommand\pythonexternal[2][]{{
\pythonstyle
\lstinputlisting[#1]{#2}}}

% Python for inline
\newcommand\pythoninline[1]{{\pythonstyle\lstinline!#1!}}

\usepackage{docmute}

\def \includeDir {src/}

\begin{document}

\include{\includeDir title}

\tableofcontents

\newpage
\section{Цель работы}
Цели учебной практики ~--- изучение математического пакета MathCad, языка 
программирования Python, математического редактора TeX.

\newpage
\section{Python}
\label{sec:python}
В первые два дня практики по языку Python <<на дом>> задавались лабораторные работы. В первый день отрабатывались основные возможности языка (массивы, срезы, assert, range и.т.д.). На второй день шла ООП-составляющая (@staticmethod, @classmethod и т.п.). В последний день была сделана консольная копия КДЗ по дисциплине <<Программирование>>.
\subsection{Лабораторная работа 1}
\subsubsection{Задание}
Выполнить задания из списка и загрузить готовое решение в LMS. Условия заданий в приложении <<\nameref{sec:attachment3}>>

\subsubsection{Результат выполнения}
Текст результата работы находится в приложении <<\nameref{sec:attachment4}>>.\\

\subsection{Лабораторная работа 2}
\subsubsection{Задание}
Выполнить задания из списка и загрузить готовое решение в LMS. 
\begin{enumerate}
	\item Файл с задачами переименовать в tasks.py
	\item Задания с предыдущего семинара разместить в отдельный
модуль (папка). Имя - ваша фамилия.
	\item В корне модуля ({\underline{\ \ }}init{\underline{\ \ }}.py) объявить свой класс исключений (или несколько), наследуясь от BaseException. Добавить в функции проверки входных данных с генерацией соответствующего ислючения. Для случаев типа "деление на ноль", добавить assertion проверки.
	\item Написать свою версию функции try parse для int и float.
	\item Создать main.py, в котором будет импортироваться ваш модуль. Организовать работу через этот файл в интерактивном режиме. Предложение списка задач, выбор задачи, ввод исходных параметров, вывод результата и так по кругу.
	\item Определить отдельный модуль (как папка с {\underline{\ \ }}init{\underline{\ \ }}.py) geom, в нем определить файл-модули vector и shape
	\item В модуле shape определить следующие классы:
Point - точка (x, y)
Shape - базовый класс, методы: периметр, прощадь, - свойства:
is{\underline{\ }}even (равносторонний ли)
Triangle (Shape) - композиция из трех точек (a, b, c), с переопределением базовых методов
Rectangle (Shape) - композиция из четырех точек (a, b, c, d), с переопределением базовых методов
\end{enumerate}

\subsubsection{Результат выполнения}
Текст результата работы находится в приложении <<\nameref{sec:attachment5}>>.\\

\subsection{Контрольное домашнее задание}
\subsubsection{Задание}
\textbf{Выполнить задание из условия и загрузить готовое решение в LMS.} \medskip\\
Предлагается разработать иерархию классов. Дается информация о базовом
абстрактном классе и его дочерних классах.\\
\indent В зависимости от варианта, может предлагаться некоторый минимальный набор атрибутов, которые необходимо реализовать. Также для каждого из предложенных классов необходимо придумать и реализовать еще минимум 3 атрибута.\\
\indent В базовом классе требуется определить абстрактный метод, описывающий задачу, решение которой конкретизируется в дочерних классах, где, соответственно, требуется написать перекрытую реализацию этого метода, использующую уникальные для каждого дочернего класса атрибуты.\\
\indent Помимо предложенного абстрактного метода базового класса необходимо придумать, описать в
базовом и реализовать в производных классах минимум еще один абстрактный метод, решающий
задачу, связанную с предметной областью.\\
\indent Каждый вариант подразумевает творческое осмысление задачи в рамках предложенной предметной
области. \bigskip\\
Классы представляют иерархию видов верховной власти. Абстрактный метод: стоимость содержания.\\
\indent Базовый класс: вид власти. Минимальный набор атрибутов: название, число членов власти.\medskip\\
Производные классы:
\begin{itemize}
	\item президентская — стоимость определяется совокупной стоимостью дворцов, резиденций, пер-
сональных а/м и самолетов, стоимостью содержания охраны, конюшни и любимой собачки;
отдельной строкой идет засекреченная статья расходов;
	\item парламентская — стоимость определяется числом парламентариев, умноженным на среднюю
стоимость на каждого: квартиры, машины, 3 помощников;
	\item королевская — стоимость определяется индивидуально (придумать).
\end{itemize}
\subsubsection{Результат выполнения}
Текст результата работы находится в приложении <<\nameref{sec:attachment6}>>.\\

\subsection{Выводы по разделу}
За 5 дней дисциплины я изучил основы языка программирования Python. Освоил основные особенности языка (и отличия от Cи-Шарпа). Научился работать с <<длинной арифметикой>>.

\newpage

\section{\TeX}
\label{sec:tex}
В качестве домашнего задания было предложено скопировать несколько страниц из научных журналов с помощью \TeX . Мной был выбран журнал \textit{<<Implementation and Applications of Automata 13th>>}. Страницы: $134, 152, 154, 205, 231, 251$.\\
\indent Вторым этапом домашней работы стало создание презентация с помощью пакета \textit{<<Beamer>>}. Тему презентации преподаватель разрешил придумать самостоятельно.  
\subsection{Репродукция статьи из журнала}
\subsubsection{Задание}
Требования:
\begin{enumerate}
	\item Таблица (не менее 4x4)
	\item Списки (нумерованные/ненумерованные)
	\item Иллюстрации (картинки с подписью)
	\item Программный код и описание алгоритма - пакеты algorithm2e или listings
	\item Сложные математические формулы (не менее 3 формул, формулы должны быть действительно сложные - дроби, сложные индексы, крышечки, интегралы, пределы итп итп)
	\item Библиографический список (не менее 5 источников)
	\item Векторная графика (на TikZ/PGF)
\end{enumerate}

\subsubsection{Результат выполнения}
\begin{enumerate}
	\item Сверстанный {\TeX}~---файл 
	\item PDF-файл с результатом работы
\end{enumerate}
Текст результата работы находится в приложении <<\nameref{sec:attachment1}>>.\medskip \\
\textbf{Все файлы из приложения находятся в соответствующих директориях данного отчета.}

\newpage
\subsection{Презентация}
\subsubsection{Задание}
Составить презентацию, используя пакет \textit{<<Beamer>>}.
Требования:
\begin{enumerate}
	\item Титульный лист
	\item Оформление (тема, логотип, и.т.п.)
	\item Overlays
	\item Картинка
	\item Список
\end{enumerate}
\subsubsection{Результат выполнения}
\begin{enumerate}
	\item Сверстанный {\TeX}~---файл 
	\item PDF-файл с результатом работы
\end{enumerate}
Текст результата работы находится в приложении <<\nameref{sec:attachment2}>>.\\

\subsection{Выводы по разделу}
В ходе учебной практики я получил навыки подготовки научных публикаций в системе \TeX . Приобрел навыки настройки рабочего окружения и организации работы в системе \TeX . Изучил основные пакеты по работе с векторной и растровой графикой.
\newpage
\section{MathCAD}
\label{sec:mathcad}
В курсе MathCAD требовалось реализовать алгоритм шифрования RSA и все необходимые для его работы алгоритмы (вычисление $h$-ричной записи $10$-ричного числа, вычисление модулярной степени $a^k (mod n)$ и прочие). Отчет по курсу должен быть сверстан в \TeX -формате. 
\subsubsection{Задание}
Предварительно подставив свои данные, провести сеанс шифрования-дешифрования сообщения и сеанс получения ЭЦП. \medskip\\ 
\textbf{Мой вариант: $p = 5903, q = 5479$}
\subsubsection{Результат выполнения}
Текст результата работы находится в приложении <<\nameref{sec:attachment7}>>.\\

\subsection{Выводы по разделу}
Во время курса по программе MathCAD я сделал свою реализацию алгоритма RSA. Получил навыки работы с программой MathCAD. Научился делать свои функции для быстро вычисления нужных мне математических выражений.
\newpage
\section{Приложения}
\subsection{Условие первого домашнего задания.}
\label{sec:attachment3}

\textbf{Здесь и далее. Формат файла не позволяет конкатенацию страниц, текст результата работы  начинается со следующей страницы.}
\includepdf[pages=-]{\includeDir python_task1.pdf}

\newpage
\subsection{Исходный код решения первого домашнего задания}
\label{sec:attachment4}

\textbf{main.py}\\
\pythonexternal{\includeDir python/task1/main.py}
\newpage
\subsection{Исходный код решения второго домашнего задания}
\label{sec:attachment5}
\textbf{main.py}\\
\pythonexternal{\includeDir python/task2/main.py}
\textbf{parser.py}\\
\pythonexternal{\includeDir python/task2/parser.py}
\textbf{shape.py}\\
\pythonexternal{\includeDir python/task2/shape.py}
\textbf{vector.py}\\
\pythonexternal{\includeDir python/task2/vector.py}
\newpage

\subsection{Исходный код КДЗ}
\label{sec:attachment6}

\textbf{main.py}\\
\pythonexternal{\includeDir python/kdz/main.py}
\textbf{governments.py}\\
\pythonexternal{\includeDir python/kdz/governments.py}
\textbf{expenses.py}\\
\pythonexternal{\includeDir python/kdz/expenses.py}



\newpage
\subsection{Репродукция статьи из журнала}
\label{sec:attachment1}
\includepdf[pages=-]{\includeDir tex.pdf}


\newpage
\subsection{Презентация с использованием пакета \textit{<<Beamer>>}}
\label{sec:attachment2}

\includepdf[pages=-]{\includeDir presentation.pdf}


\newpage
\subsection{Решение MathCAD}
\label{sec:attachment7}

\includepdf[pages=-]{\includeDir mathcad.pdf}

\end{document}
