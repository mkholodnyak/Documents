\documentclass[a4paper,12pt]{report} % формат бумаги А4, шрифт по умолчанию - 12pt

\usepackage[T2A]{fontenc} % поддержка кириллицы в Латехе
\usepackage[utf8]{inputenc} % включаю кодировку ютф8
\usepackage[english,russian]{babel} % использую русский и английский языки с переносами

\usepackage{indentfirst} % делать отступ в начале параграфа
\usepackage{amsmath} % математические штуковины
\usepackage{mathtools} % еще математические штуковины
\usepackage{mathtext}
\usepackage{multicol} % подключаю мультиколоночность в тексте
\usepackage{graphicx} % пакет для вставки графики, я хз зачем он нужен в этом документе
\usepackage{listings} % пакет для вставки кода


\usepackage{geometry} % меняю поля страницы

%из параметров ниже понятно, какие части полей страницы меняются:
\geometry{left=2.5cm}
\geometry{right=1cm}
\geometry{top=2cm}
\geometry{bottom=2cm}

\renewcommand{\baselinestretch}{1.5} % меняю ширину между строками на 1.5
\righthyphenmin=2

\begin{document}

\begin{titlepage}
\newpage

\begin{center}
{\large НАЦИОНАЛЬНЫЙ ИССЛЕДОВАТЕЛЬСКИЙ УНИВЕРСИТЕТ \\
«ВЫСШАЯ ШКОЛА ЭКОНОМИКИ» 							\\
Дисциплина: «Дискретная математика. Алгоритмический подход.»}

\vfill % заполняет длину страницы вертикально

{\large Домашнее задание 1}

\bigskip


{\large \underline{Функции алгебры логики}}\\
Вариант 1

\vfill

\begin{flushright}
Выполнил: Холодняк М.В.,\\
студент группы 174ПИ.\medskip \\
Преподаватель: Броневич А.Г.
\end{flushright}

\vfill

Москва \number\year \\
\LaTeX

\end{center}
\end{titlepage}

\newpage

\begin{center}
Задание 1.\\
\end{center}

\begin{flushleft}
\begin{multicols}{2}
\begin{tabular}{ | c | c | c | c |}
\hline
X & Y & Z & $(x \implies y) \wedge z \implies (x \vee y)$ \\
\hline
0 & 0 & 0 & 1 \\
\hline
0 & 0 & 1 & 0 \\ 
\hline
0 & 1 & 0 & 1 \\
\hline
0 & 1 & 1 & 1 \\
\hline
1 & 0 & 0 & 1 \\
\hline
1 & 0 & 1 & 1 \\
\hline
1 & 1 & 0 & 1 \\
\hline
1 & 1 & 1 & 1 \\
\hline
\end{tabular} \\

\begin{tabular}{ | c | c | c | c |}
\hline
X & Y & Z & $((x \oplus y) \wedge z \implies (z \vee y)) \equiv (x \vee y)$ \\
\hline
0 & 0 & 0 & 0 \\
\hline
0 & 0 & 1 & 0 \\ 
\hline
0 & 1 & 0 & 1 \\
\hline
0 & 1 & 1 & 1 \\
\hline
1 & 0 & 0 & 1 \\
\hline
1 & 0 & 1 & 1 \\
\hline
1 & 1 & 0 & 1 \\
\hline
1 & 1 & 1 & 1 \\
\hline
\end{tabular} \\
\end{multicols}
\bigskip

\begin{center}
\begin{tabular}{ | c | c | c | c |}
\hline
X & Y & Z & $\overline{(x \oplus y) \wedge ((x \equiv y) \implies z) \vee y}$ \\
\hline
0 & 0 & 0 & 1 \\
\hline
0 & 0 & 1 & 1 \\ 
\hline
0 & 1 & 0 & 0 \\
\hline
0 & 1 & 1 & 0 \\
\hline
1 & 0 & 0 & 0 \\
\hline
1 & 0 & 1 & 0 \\
\hline
1 & 1 & 0 & 0 \\
\hline
1 & 1 & 1 & 0 \\
\hline
\end{tabular} \\
\end{center}

Функция $F_1(X,Y,Z)$ ложна при $F_1(0,0,1)$\\
Функция $F_2(X,Y,Z)$ ложна при $F_2(0,0,0), F_2(0,0,1)$\\
Функция $F_3(X,Y,Z)$ ложна при $F_3(0,1,0), F_3(0,1,1), F_3(1,0,0), F_3(1,0,1), F_3(1,1,0), F_3(1,1,1)$\\

\end{flushleft}

\bigskip


\begin{center}
Задание  2.\\
\end{center}

\begin{flushleft}


\end{flushleft}

\bigskip

       
\begin{center}
Задание 3.
\end{center}

\begin{flushleft}
Дизъюнтивные нормальные формы:\\
$F_1(X,Y,Z)=\overline{XYZ} \vee \overline{X}Y\bar{Z} \vee \overline{X}YZ \vee X\overline{YZ} \vee X\overline{Y}Z \vee XY\overline{Z} \vee XYZ$\\
$F_2(X,Y,Z)=\overline{X}Y\overline{Z} \vee \overline{X}YZ \vee X\overline{YZ} \vee X\overline{Y}Z \vee XY\overline{Z} \vee XYZ$\\
$F_3(X,Y,Z)=\overline{XYZ} \vee \overline{XY}Z$\\
\bigskip
Конъюктивные нормальные формы:\\
$F_1(X,Y,Z)=X \vee Y \vee \overline{Z}$\\
$F_2(X,Y,Z)=(X\vee Y\vee Z)(X\vee Y\vee\overline{Z})$\\
$F_3(X,Y,Z)=(X\vee \overline{Y}\vee Z)(X \vee \overline{Y} \vee \overline{Z})(\overline{X} \vee Y \vee Z)(\overline{X} \vee Y \vee \overline{Z})(\overline{X} \vee \overline{Y} \vee Z)(\overline{X} \vee \overline{Y} \vee \overline{Z})$\\
\end{flushleft}

\bigskip


\begin{center}
Задание 4.\\
\end{center}

\begin{flushleft}
Полиномы Жегалкина:\\
$F_1(X,Y,Z)=1 \oplus Z \oplus YZ \oplus XZ \oplus XYZ$\\
$F_2(X,Y,Z)=Y \oplus X \oplus XY$\\
$F_3(X,Y,Z)=1 \oplus Y \oplus X \oplus XY$\\
\end{flushleft}

\bigskip

\begin{center}
Задание 5.\\
\end{center}

\begin{flushleft}

\end{flushleft}

\bigskip

\begin{center}
Задание 6.\\
\end{center}

\begin{flushleft}

\end{flushleft}

\bigskip

\begin{center}
Задание 7.\\
\end{center}

\begin{flushleft}
Условие двойственности функций:\\
Функция $f^{*}(x_1, \ldots , x_n)$ называется \textbf{двойственной} к функции $f(x_1, \ldots , x_n)$,\\
если $f^{*}(x_1, \ldots , x_n) = \overline{f}(\overline{x_1}, \ldots , \overline{x_n})$\\
\bigskip
$f=(A \vee B) \wedge (A \vee C) \wedge (B \vee D) \wedge (C \vee D)$\\
$g= (A \vee D) \wedge (B \vee C)$\\
Согласно принципу двойственности имеем:\\
$f^{*}= AB \vee AC \vee BD \vee CD = A(B \vee C) \vee D(B \vee C) = (A\vee D)(B \vee C)$\\
$f^{*} = g$\\
Следовательно, функции являются двойственными.
\end{flushleft}

\bigskip

\begin{center}
Задание 8.\\
\end{center}

\begin{flushleft}

\end{flushleft}

\bigskip

\begin{center}
Задание 9.\\
\end{center}

\begin{flushleft}

\end{flushleft}
\end{document}
