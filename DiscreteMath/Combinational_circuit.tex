\documentclass[a4paper,12pt]{report} % формат бумаги А4, шрифт по умолчанию - 12pt

% заметь, что в квадратных скобках вводятся необязательные аргументы пакетов.
% а в фигурных - обязательные

\usepackage[T2A]{fontenc} % поддержка кириллицы в Латехе
\usepackage[utf8]{inputenc} % включаю кодировку ютф8
\usepackage[english,russian]{babel} % использую русский и английский языки с переносами

\usepackage{indentfirst} % делать отступ в начале параграфа
\usepackage{amsmath} % математические штуковины
\usepackage{mathtools} % еще математические штуковины
\usepackage{mathtext}
\usepackage{multicol} % подключаю мультиколоночность в тексте
\usepackage{graphicx} % пакет для вставки графики, я хз нахуя он нужен в этом документе
\usepackage{listings} % пакет для вставки кода


\usepackage{geometry} % меняю поля страницы

%из параметров ниже понятно, какие части полей страницы меняются:
\geometry{left=2.5cm}
\geometry{right=1cm}
\geometry{top=2cm}
\geometry{bottom=2cm}

\renewcommand{\baselinestretch}{1} % меняю ширину между строками на 1.5
\righthyphenmin=2
\begin{document}

\begin{titlepage}
\newpage

\begin{center}
{\large НАЦИОНАЛЬНЫЙ ИССЛЕДОВАТЕЛЬСКИЙ УНИВЕРСИТЕТ \\
«ВЫСШАЯ ШКОЛА ЭКОНОМИКИ» 							\\
Дисциплина: «Информатика, математическая логика и теория алгоритмов»}

\vfill % заполняет длину страницы вертикально

{\large Домашнее задание 1}

\bigskip

\underline{Исследование комбинационных схем}\\
Вариант 113

\vfill

\begin{flushright}
Выполнил: Холодняк Максим,\\
студент группы 174ПИ.\medskip \\
Преподаватель: Авдошин С.М.,\\
профессор кафедры УРПО \\
отделения программной инженерии \\
факультета бизнес-информатики \\
\end{flushright}

\vfill

Москва \number\year

\end{center}
\end{titlepage}

\newpage

\begin{center}
№1.\\
\end{center}


\begin{flushleft}
$252X_7 + 135X_6 +232X_5+114X_4 +79X_3+242X_2+224X_1+49X_0 = 40$ \\
Переведём коэффициенты уравнения в двоичную систему счисления.

$252_2 = 1111\,1100$ , $135_2 = 1000\,0111$ , $232_2 = 1110\,1000$ , $114_2 = 0111\, 0010$,\\
$79_2 = 0100\,1111$ , $242_2 = 1111\, 0010$ , $224_2 = 1110\,0000$ , $49_2 = 0011\, 0001$ , $40_2 = 0010\,1000$.\\

Составим расширенную матрицу коэффициентов соответствующей системы линейных уравнений в GF(2) и решим систему. \\

\bigskip
\bigskip

% определяю новую матрицу с чертой
\newenvironment{amatrix}[1]{
	\left(\begin{array}{@{}*{#1}{c}|c@{}}
	}{
	\end{array}\right)
	}
%
\scriptsize{
$	\begin{smallmatrix}
\, & \,
	\end{smallmatrix}
$
$\begin{amatrix}{8}
\fbox{1}&1&1&0&0&1&1&0&0\\
1&0&1&1&1&1&1&0&0\\
1&0&1&1&0&1&1&1&1\\
1&0&0&1&0&1&0&1&0\\
1&0&1&0&1&0&0&0&1\\
1&1&0&0&1&0&0&0&0\\
0&1&0&1&1&1&0&0&0\\
0&1&0&0&1&0&0&1&0\\
\end{amatrix}$
$	\begin{matrix}
	(2)\oplus=(1)\\
	(3)\oplus=(1)\\
	(4)\oplus=(1)\\
	(5)\oplus=(1)\\
	(6)\oplus=(1)\\
	\sim
	\end{matrix}
$
$
\begin{amatrix}{8}
1&1&1&0&0&1&1&0&0\\
0&\fbox{1}&0&1&1&0&0&0&0\\
0&1&0&1&0&0&0&1&1\\
0&1&1&1&0&0&1&1&0\\
0&1&0&0&1&1&1&0&1\\
0&0&1&0&1&1&1&0&0\\
0&1&0&1&1&1&0&0&0\\
0&1&0&0&1&0&0&1&0
\end{amatrix}$
$	\begin{matrix}
	(1)\oplus=(2)\\
	(3)\oplus=(2)\\
	(4)\oplus=(2)\\
	(5)\oplus=(2)\\
	(7)\oplus=(2)\\
	(8)\oplus=(2)\\
	\sim
	\end{matrix}
$
\\
\bigskip
$	\begin{matrix}
	\sim
	\end{matrix}
$
$\begin{amatrix}{8}
1&0&1&1&1&1&1&0&0\\
0&1&0&1&1&0&0&0&0\\
0&0&0&0&1&0&0&1&1\\
0&0&\fbox{1}&0&1&0&1&1&0\\
0&0&0&1&0&1&1&0&1\\
0&0&1&0&1&1&1&0&0\\
0&0&0&0&0&1&0&0&0\\
0&0&0&1&0&0&0&1&0\\
\end{amatrix}$
$	\begin{matrix}
	(1)\oplus=(4)\\
	(6)\oplus=(4)\\
	\sim
	\end{matrix}
$
$\begin{amatrix}{8}
1&0&0&1&0&1&0&1&0\\
0&1&0&1&1&0&0&0&0\\
0&0&0&0&1&0&0&1&1\\
0&0&1&0&1&0&1&1&0\\
0&0&0&\fbox{1}&0&1&1&0&1\\
0&0&0&0&0&1&0&1&0\\
0&0&0&0&0&1&0&0&0\\
0&0&0&1&0&0&0&1&0
\end{amatrix}$
$	\begin{matrix}
	(1)\oplus=(5)\\
	(2)\oplus=(5)\\
	(8)\oplus=(5)\\
	\sim
	\end{matrix}
$
\\
\bigskip
$	\begin{matrix}
	\sim
	\end{matrix}
$
$\begin{amatrix}{8}
1&0&0&0&0&0&1&1&1\\
0&1&0&0&1&1&1&0&1\\
0&0&0&0&\fbox{1}&0&0&1&1\\
0&0&1&0&1&0&1&1&0\\
0&0&0&1&0&1&1&0&1\\
0&0&0&0&0&1&0&1&0\\
0&0&0&0&0&1&0&0&0\\
0&0&0&0&0&1&1&1&1
\end{amatrix}$
$	\begin{matrix}
	(2)\oplus=(3)\\
	(4)\oplus=(3)\\
	\sim
	\end{matrix}
$
$\begin{amatrix}{8}
1&0&0&0&0&0&1&1&1\\
0&1&0&0&0&1&1&1&0\\
0&0&0&0&1&0&0&1&1\\
0&0&1&0&0&0&1&0&1\\
0&0&0&1&0&1&1&0&1\\
0&0&0&0&0&\fbox{1}&0&1&0\\
0&0&0&0&0&1&0&0&0\\
0&0&0&0&0&1&1&1&1
\end{amatrix}$
$	\begin{matrix}
	(2)\oplus=(6)\\
	(5)\oplus=(6)\\
	(7)\oplus=(6)\\
	(8)\oplus=(6)\\
	\sim
	\end{matrix}
$
\\
\bigskip
$	\begin{matrix}
	\sim
	\end{matrix}
$
$\begin{amatrix}{8}
1&0&0&0&0&0&1&1&1\\
0&1&0&0&0&0&1&0&0\\
0&0&0&0&1&0&0&1&1\\
0&0&1&0&0&0&1&0&1\\
0&0&0&1&0&0&1&1&1\\
0&0&0&0&0&1&0&1&0\\
0&0&0&0&0&0&0&1&0\\
0&0&0&0&0&0&\fbox{1}&0&1
\end{amatrix}$
$	\begin{matrix}
	(1)\oplus=(8)\\
	(2)\oplus=(8)\\
	(4)\oplus=(8)\\
	(5)\oplus=(8)\\
	\sim
	\end{matrix}
$
$\begin{amatrix}{8}
1&0&0&0&0&0&0&1&0\\
0&1&0&0&0&0&0&0&1\\
0&0&0&0&1&0&0&1&1\\
0&0&1&0&0&0&0&0&0\\
0&0&0&1&0&0&0&1&0\\
0&0&0&0&0&1&0&1&0\\
0&0&0&0&0&0&0&\fbox{1}&0\\
0&0&0&0&0&0&1&0&1
\end{amatrix}$
$	\begin{matrix}
	(1)\oplus=(7)\\
	(3)\oplus=(7)\\
	(5)\oplus=(7)\\
	(6)\oplus=(7)\\
	\sim
	\end{matrix}
$
\\
\bigskip
$	\begin{matrix}
	\sim
	\end{matrix}
$
$\begin{amatrix}{8}
1&0&0&0&0&0&0&0&0\\
0&1&0&0&0&0&0&0&1\\
0&0&0&0&1&0&0&0&1\\
0&0&1&0&0&0&0&0&0\\
0&0&0&1&0&0&0&0&0\\
0&0&0&0&0&1&0&0&0\\
0&0&0&0&0&0&0&1&0\\
0&0&0&0&0&0&1&0&1
\end{amatrix}$

}
\end{flushleft}

\bigskip

\begin{flushleft}
В описаниях преобразований строки обозначены как (1), (2), ..., (8), а выражение $ (i)\oplus{=(j)} $ обозначает «заменить все числа в строке ($i$) на их сумму по модулю $2$ с соответствующими числами строки ($j$)». \\
Получаем решение: $X_7$ = 0, $X_6$ = 1, $X_5$ = 0, $X_4$ = 0, $X_3$ = 1, $X_2$ = 0, $X_1$ = 1, $X_0$ = 0. 
\end{flushleft}

\newpage
\begin{flushleft}
Составим таблицу истинности функции F. \\

\bigskip

\begin{tabular}{|c|c|c|c|c|c|c|c|c|}
\hline
A &0 &0 &0 &0 &1 &1 &1 &1 \\
\hline
B &0 &0 &1 &1 &0 &0 &1 &1 \\
\hline
C &0 &1 &0 &1 &0 &1 &0 &1 \\
\hline
F &0 & 1 & 0 & 0 & 1 & 0 & 1 & 0 \\
\hline
\end{tabular}
\\
\bigskip

Десятичный номер функции F равен $2^1 + 2^3 + 2^6 = 74$.
\end{flushleft}

\begin{multicols}{2}



\bigskip
\centering
№4.\\
\bigskip
\begin{tabular}{|c|c|c|c|c|c|c|c|c|}
\hline
A &0 &0 &0 &0 &1 &1 &1 &1 \\
\hline
B &0 &0 &1 &1 &0 &0 &1 &1 \\
\hline
C &0 &1 &0 &1 &0 &1 &0 &1 \\
\hline
F &0 & 1 & 0 & 0 & 1 &0&1&0 \\
\hline
$F_A^{\prime}$ &1&1&1&0&1&1&1&0\\
\hline
$F_B^{\prime}$ &0&1&0&1&0&0&0&0\\
\hline
$F_C^{\prime}$ & 1&1&0&0&1&1&1&1\\
\hline
$F_{A,B}^{\prime\prime}$ &0&1&0&1&0&1&0&1\\
\hline
$F_{B,C}^{\prime\prime}$ &1&1&1&1&0&0&0&0\\
\hline
$F_{A,C}^{\prime\prime}$ &0&0&1&1&0&0&1&1\\
\hline
$F_{A,B,C}^{\prime\prime\prime}$ &1&1&1&1&1&1&1&1\\

\hline
\end{tabular}\\
\bigskip

№6.\\
\bigskip
\begin{tabular}{|c|c|c|c|c|c|c|c|c|}
\hline
A &0 &0 &0 &0 &1 &1 &1 &1 \\
\hline
B &0 &0 &1 &1 &0 &0 &1 &1 \\
\hline
C &0 &1 &0 &1 &0 &1 &0 &1 \\
\hline
F &0 & 1 & 0 & 0 & 1 &0&1&0 \\
\hline
$F_{(A,B)}^{\prime}$ & 1&1&1&0&1&0&1&1\\
\hline
$F_{(B,C)}^{\prime}$ & 0&1&1&0&1&1&1&1\\
\hline
$F_{(A,C)}^{\prime}$ & 0&0&0&1&0&0&1&0\\
\hline
\end{tabular}
\\
\bigskip
№8.\\
\bigskip
\begin{tabular}{|c|c|c|c|c|c|c|c|c|}
\hline
A &0 &0 &0 &0 &1 &1 &1 &1 \\
\hline
B &0 &0 &1 &1 &0 &0 &1 &1 \\
\hline
C &0 &1 &0 &1 &0 &1 &0 &1 \\
\hline
F &0 & 1 & 0 & 0 & 1 & 0 & 1 & 0 \\
\hline
$F_{(A,B,C)}^{\prime}$ & 0&0&0&1&1&0&0&0\\
\hline
\end{tabular}\\
\bigskip
\columnbreak
№5. \\
% РЕШЕНО
\bigskip
$ F_A^{\prime} = \overline{BC} $ \\
$ F_B^{\prime} = \overline{A}C $ \\
$ F_C^{\prime} = (B \Rightarrow A $) \\
$ F_{A,B}^{\prime\prime} =  C$ \\
$ F_{B,C}^{\prime\prime} =  \overline{A}$ \\
$ F_{A,C}^{\prime\prime} =  B$ \\
$ F_{A,B,C}^{\prime\prime\prime} = 1 $ \\
\bigskip
\bigskip
№7. \\
% Решено
\bigskip
$ F_{(A,B)}^{\prime} =  (C \Rightarrow (A \equiv B))$ \\
$ F_{(B,C)}^{\prime} =  (B \equiv C) \Rightarrow A$ \\
$ F_{(A,C)}^{\prime} = B(A \oplus C)$ \\
\bigskip
\bigskip
№9. \\
% Решено
\bigskip
$ F_{(A,B,C)}^{\prime} = (\overline{A} \equiv B \equiv C) $ \\
\end{multicols}

\newpage
\begin{center}
№10.
\end{center}
\begin{flushleft}
$ F(A,B,C)  = A \oplus C \oplus BC \oplus ABC$ 
\end{flushleft}
\bigskip
\begin{center}
№11.
\end{center}
$ (0,0,0): F(A,B,C) = A \oplus C \oplus BC \oplus ABC$  \\
$ (0,0,1): F(A,B,C) = 1 \oplus A \oplus B \oplus (C\oplus 1)\oplus AB \oplus B(C\oplus 1) \oplus AB(C\oplus1) $ \\
$ (0,1,0): F(A,B,C) = A \oplus (B\oplus 1)C \oplus AC \oplus A(B\oplus 1)C$ \\
$ (0,1,1): F(A,B,C) = (B\oplus 1) \oplus A(B \oplus 1) \oplus (B \oplus 1)(C \oplus 1) \oplus A(C \oplus 1) \oplus A(B\oplus 1)(C \oplus 1) $ \\
$ (1,0,0): F(A,B,C) = 1 \oplus (A \oplus 1) \oplus C \oplus (A \oplus 1)BC $ \\
$ (1,0,1): F(A,B,C) = (A \oplus 1) \oplus (C \oplus 1) \oplus (A \oplus 1)B \oplus (A \oplus 1)B(C \oplus 1) $ \\
$ (1,1,0): F(A,B,C) = 1 \oplus (A \oplus 1) \oplus C \oplus (A \oplus 1)C \oplus (A \oplus 1)(B \oplus 1)C$\\
$ (1,1,1): F(A,B,C) = (C \oplus 1) \oplus (A \oplus 1)(B \oplus 1)\oplus (A \oplus 1)(C \oplus 1) \oplus (A \oplus 1)(B \oplus 1)(C \oplus 1) $ \\
\begin{center} 
№12.  
% Решено
\end{center} 
\begin{flushleft}
$ F(A,B,C) = 0 \equiv (A \equiv 0) \equiv (C \equiv 0) \equiv ((B \equiv 0)+(C \equiv 0)) \equiv ((A \equiv 0)+(B \equiv 0) + (C \equiv 0))$ \\
\end{flushleft}
\begin{center} 
№13.  
% Решено
\end{center} 
\begin{flushleft}
$ (0,0,0): F(A,B,C) = 0 \equiv (A \equiv 0) \equiv (C \equiv 0) \equiv ((B \equiv 0)+(C \equiv 0)) \equiv ((A \equiv 0)+(B \equiv 0) + (C \equiv 0))$ \\
$ (0,0,1): F(A,B,C) = (A \equiv 0) \equiv (B \equiv 0) \equiv C \equiv ((A \equiv 0)+(B \equiv 0)) \equiv ((B \equiv 0)+ C) \equiv((A \equiv 0) + (B \equiv 0) + C) $ \\
$ (0,1,0): F(A,B,C) = 0 \equiv (A \equiv 0) \equiv (B + (C \equiv 0)) \equiv ((A \equiv 0) + (C \equiv 0)) \equiv ((A \equiv 0) + B + (C \equiv 0))$ \\
$ (0,1,1): F(A,B,C) = 0 \equiv B \equiv ((A \equiv 0)+B)\equiv (B + C)\equiv((A \equiv 0) + C) \equiv ((A \equiv 0)+B+C)$ \\
$ (1,0,0): F(A,B,C) = A \equiv (C \equiv 0) \equiv (A + (B \equiv 0)+ (C \equiv 0))$ \\
$ (1,0,1): F(A,B,C) =0 \equiv A \equiv C \equiv(A + (B \equiv 0)) \equiv (A + (B \equiv 0)+C)$ \\
$ (1,1,0): F(A,B,C) = A \equiv (C \equiv 0) \equiv (A + (C \equiv0)) \equiv (A + B+(C \equiv 0))$ \\
$ (1,1,1): F(A,B,C) = 0 \equiv C \equiv (A+ B)\equiv (A \equiv C) \equiv (A+B+C)$ \\
\end{flushleft}
\begin{center} 
№14.  
% Решено
\end{center}
\begin{flushleft}
$F \in{T_0}, \text{т.к.}\; F(0,0,0)=0 $ \\
$F \notin{T_1},\text{т.к.}\; F(1,1,1)=0 $ \\
$F \notin{T_\leq}, \text{т.к.}\; F(0,0,1)>F(0,1,0) $ \\
$F \notin{T_*}, \text{т.к.}\; F(1,1,1)=F(0,0,0) = 0 $ \\
$F \notin{T_L}, \text{т.к.}\; F(0,0,0)= A \oplus C \oplus BC \oplus ABC $ \\
\end{flushleft}
\begin{center} 
№15.  
\end{center}
$0=F(A,A,A)$\\
$A\oplus B = F(A,A,B)$\\
$A = F(A, A, F(A,A,A))$\\
$B = F(F(A,A,A),F(A,A,A),B)$\\
\end{document}
